\section{Einleitung}
In diesem Versuch wird das Verhalten von polarisiertem Licht untersucht.
\subsection{Polarisation durch Reflexion}
Es wird von linear polarisiertem Licht gesprochen, wenn der E-Feldvektor in nur einer Raumrichtung schwingt und von zirkular polarisiertem Licht wenn der E-Feldvektor eine Schraubenlinie beschreibt. Es wird elliptisch polarisiert genannt, wenn der E-Feldvektor eine Ellipse beschreibt.
Trifft Licht auf eine Glasfläche mit Brechungsindex $ n $ wird der Strahl teilweise reflektiert und teilweise gebrochen. Der Anteil des reflektierten Strahles hängt vom Einfallswinkel $ \alpha $ und von der Polarisation des Lichts ab.
Dabei hat senkrecht zur Einfallsebene Polarisiertes Licht (s-polarisiert) ein größeres Reflexionsvermögen als parallel zur Einfallsebene Polarisiertes Licht (p-polarisiert).
Dabei gibt es für p-polarisiertes Licht einen Einfallswinkel $ \alpha_{B} $ bei welchem kein Licht reflektiert wird, es gilt
\begin{equation}
\tan \alpha_{B} = n
\end{equation}
somit ist das reflektierte Licht vollständig linear s-polarisiert.
\subsection{Doppelbrechung in Kalkspat}
Beim Durchgang von unpolarisierten Licht durch den Kalkspatkristall, wird dieser in zwei Teilstrahlen aufgeteilt.
Ein Strahl verhält sich nach dem Snelliusschen Brechungsgesetz und wird daher auch ordentlicher Strahl genannt, für diesen gilt der Brechungindex $ n_{O} $. Der andere Teilstrahl wird außerordentlicher Strahl genannt, für diesen gilt der Brechungsindex $ n_{a} $. Die beiden Teilstrahlen sind orthogonal zueinander polarisiert. Der ordentliche Strahl ist s-polarisiert und der außerordentliche Strahl ist p-polarisiert.
\subsection{Polarisation durch selektive Absorption}
Polarisator und Analysator sind zwei einfache Polarisationsfilter. F¨allt unpolarisiertes
Licht auf so einen Filter, wird dieser Polarisator genannt. Mit einem
zweiten Filter, dem Analysator, kann man nun untersuchen, ob z.B. die Polarisationsebene
sich gedreht hat. Die Komponente $\vec E = \vec E_0 \cos\varphi$ wird durchgelassen und die Komponente $\vec E' = \vec E_0 \cos\varphi$ wird absorbiert, wenn $\varphi$ der Winkel zwischen den Filtern ist. Da der gemessene Kurzschlussstrom eines Photoelements proportional der Lichtintensität I ist, folgt
\begin{equation}
I = I_0 \cos^2\varphi
\end{equation}
\subsection{\lambda/2-Platte}
Häufig ist es von Vorteil die Schwingungsebene von linear polarisiertem Licht um einen Winkel $ \bigtriangleup\alpha $ zu drehen. Mit einer $ \lambda/2-Platte $ ist dies möglich. Wenn der $ \vec{E}-Vektor $ den Winkel $ \varphi $ gegen die optische Achse hat, so kann dieser in zwei Komponenten aufspalte. Diese sind einmal der parallele Anteil $ E_{\|} = E\cos \varphi $ und der senkrechte Anteil $ E_{\bot}=E\sin \varphi $ . Treffen die beiden Komponente in die Platte ein sind sie in Phase. Die Komponenten haben unterschiedliche Laufzeiten, so entsteht zwischen ihnen eine Phasendifferenz $\bigtriangleup\varphi $. Es gilt, 
\begin{equation}
\bigtriangleup\varphi=\frac{2\pi}{\lambda}d(n_{2}-n_{1})
\end{equation}
$d$ ist die Strecke die durchlaufen wird. Bei der $ \lambda/2 - Platte $ gilt, $ d(n_{2}-n_{1})=\lambda/2 $, also $ \bigtriangleup\varphi=\pi $. Dabei hat sich der $ \vec{E}-Vektor $ um den Winkel $\bigtriangleup=2\varphi $ Durch das drehen der $ \lambda/2-Platte $ lässt sich jeder Winkel und somit jede gewünschte Drehung herbeiführen.
\subsection{Optische Aktivität}
Diese Medien, z.B. eine Zuckerlösung, haben die Eigenschaft die Polarisationsebene linear polarisierten Lichts zu drehen. Generell fehlen solchen Substanzen bestimmte Symmetrieeigenschaften - Kristalle ohne Symmetriezentrum, Moleküle ohne Symmetriezentrum und Ebene, in allen Zustandsformen. Die Kristalle oder Moleküle sind in einer Lösung zwar statistisch verteilt, zeigen diese Eigenschaften aber trotzdem: Die Lichtwelle wird von der Substanz in eine Richtung gedreht, aber nicht in die andere Richtung zurück, da die Substanz "'auf der Rückseite`` nicht die selben Eigenschaften besitzt, da es nicht symmetrisch ist. Durchdringt eine Lichtwelle nun so eine Substanz, die in Lösung ist, so hängt der Drehwinkel $\alpha$ von der Konzentration $c$ und der durchdrungenen Länge $l$ proportional ab:
\begin{equation}
\alpha = \alpha_{s}\cdot c\cdot l
\end{equation}
$\alpha_{s}$ ist der Proportionalitätsfaktor und heißt spezifische Drehung.