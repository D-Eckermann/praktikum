% ################################################################
% #                                                              #
% # Autor: Michael Epping                                        #
% # E-Mail: michael.epping@uni-muenster.de                       #
% # Version: 1.4                                                 #
% # Datum: Juni 2013                                             #
% # Info: Diese Datei sollte nicht verändert werden.             #
% #    Hier werden die Einstellungen festgelegt und              #
% #    Pakete eingebunden. Alles weitere wird über               #
% #    die Dateien verändert, die mit "0X_" beginnen.            #
% # Copyright: CC0 (macht mit diesen Dateien was ihr wollt)      #
% #    https://creativecommons.org/publicdomain/zero/1.0/deed.de #
% #                                                              #
% ################################################################

% Änderungen 1.2 -> 1.3
% * Bei der Verwendung von texlive2012 gibt es Probleme mit myalphadin.
%   Diese Vorlage für Einträge insLiteraturverzeichnis habe ich durch unsrtdin ersetzt.
% * Da ich jetzt mit TeXlipse arbeite, habe ich ein paar Anpassungen vorgenommen.
%   So ist z.B. der Name der bib-Datei fest vorgegeben, damit auch die Autovervollständigung bei BibTeX-Keys funktioniert.
% * Zusätzliche Kommentare sollten das Arbeiten mit dieser Vorlage erleichtern.
% * Die Vorlage enthält sinnvollen Text und nicht nur nutzlose Platzhalter.

% Änderungen 1.3 -> 1.4
% * Das Paket "ifthenx" gibt es unter Ubuntu 12.04 mit texlive 2009-15 nicht. Als alternative habe ich "xifthen" eingetragen.
% * Tabulatoren habe ich durch Leerzeichen ersetzt. Dadurch bleibt das Layout (Einrücken und Position Kommentare) erhalten, 
%   egal mit welchem Editor man die Dateien öffnet.
% * Ich habe einen Copyright-Vermerk hinzugefügt, nämlich dass es im Prizip keines gibt.
% * Neuer Abschnitt: latexmk
% * Neuer Abschnitt: Verbatim
% * Die Bilddatei "titelseite.jpg" wurde entfernt (wegen Copyright), da die Vorlage ab jetzt öffentlich zugänglich sein soll.
% * "README.txt" im Verzeichnis Bilder wurde erstellt.
% * Literaturangabe der Anleitung zur Optik, Wärmelehre und Atomphysik hinzugefügt. 

% ###############
% # Allgemeines #
% ###############

% Zeilen, die mit einem Prozentzeichen beginnen sind Kommentare. 
% Alle verwendeten Funktionen sind mit solchen Kommentaren versehen, so dass man den Zweck der jeweiligen Funktion nachvollziehen kann.

% ######################################
% # Konfigurieren der Dokumentenklasse #
% ######################################

\documentclass[
    a4paper,                                               % Papierformat
    oneside,                                               % Einseitig
    %twoside,                                              % Zweiseitig
    12pt,                                                  % Schriftgröße
    pagesize=auto,                                         % schreibt die Papiergröße korrekt ins Ausgabedokument
    headsepline,                                           % Linie unter der Kopfzeile
    %draft=true                                            % Markiert zu lange und zu kurze Zeilen
 ]{scrartcl}
% Es gibt die Dokumenttypen scrartcle, srcbook, scrreprt und scrlettr. Diese gehören zum KOME-Skript und sollten für deutsche Texte benutzt werden.
% Für englische Texte wählt man entsprechend article, book, report und letter.
% Es ist  nicht unbedingt zu empfehlen, bei einem bestehendem Dokument, die documentclass zu ändern.

% ####################
% # Pakete einbinden #
% ####################

% Pakete erweitern LaTeX um zusätzliche Funktionen. Dies ist eine Satz nützlicher Pakete.
% Weitere sollten in der Datei"`01_EigenePakete.tex"' hinzugefügt werden.
\usepackage[utf8x]{inputenc}                                                  % Legt die Zeichenkodierung fest, z.B UTF8
\usepackage[T1]{fontenc}                                                      % Verwendung der Zeichentabelle T1, für deutschsprachige Dokumente sinnvoll
\usepackage[ngerman,english]{babel}                                           % Silbentrennung nach neuer deutscher und englischer Rechtschreibung
\usepackage{amsmath}                                                          % Mathepaket
\usepackage{xifthen}                                                          % Wird benötigt um \ifthenelse zu benutzen
\usepackage[pdftex]{graphicx}                                                 % Zum flexiblen Einbinden von Grafiken, pdftex ist optional
\usepackage{units}                                                            % Ermöglicht die Nutzung von \unit[Zahl]{Einheit}
\usepackage{setspace}                                                         % Einfaches wechseln zwischen unterschiedlichen Zeilenabständen
\usepackage[pdfpagelabels]{hyperref}                                          % Verlinkt Textstellen im PDF Dokument
\usepackage[font=small,labelfont=bf,labelsep=endash,format=plain]{caption}    % Darstellung für Caption s.u.
\usepackage{subfig}                                                           % Bilder nebeneinander
\usepackage{wrapfig}                                                          % Fließtext um Figure-Umgebung
\usepackage{cite}                                                             % Zusatzfunktionen zum zitieren
\usepackage{scrpage2}                                                         % Wird für Kopf- und Fußzeile benötigt
\usepackage{array,dcolumn}                                                    % Beide Pakete werden für die Ausrichtung der Tabellenspalten benötigt
\usepackage{siunitx}

% weitere Pakete einbinden
% Die Folgenden Pakete sind schon eingebunden (siehe 00_Protokoll.tex):
% \usepackage[utf8x]{inputenc}                             % Legt die Zeichenkodierung fest, z.B UTF8
% \usepackage[T1]{fontenc}                                 % Verwendung der Zeichentabelle T1, für deutschsprachige Dokumente sinnvoll
% \usepackage[ngerman,english]{babel}                      % Silbentrennung nach neuer deutscher und englischer Rechtschreibung
% \usepackage{amsmath}                                     % Mathepaket
% \usepackage{ifthenx}                                     % Wird benötigt um \ifthenelse zu benutzen
% \usepackage[pdftex]{graphicx}                            % Zum flexiblen Einbinden von Grafiken, pdftex ist optional
% \usepackage{units}                                       % Ermöglicht die Nutzung von \unit[Zahl]{Einheit}
% \usepackage{setspace}                                    % Einfaches wechseln zwischen unterschiedlichen Zeilenabständen
% \usepackage[pdfpagelabels]{hyperref}                     % Verlinkt Textstellen im PDF Dokument
% \usepackage[font=small,labelfont=bf,labelsep=endash,format=plain]{caption}
%                                                          % Darstellung für Caption s.u.
% \usepackage{subfig}                                      % Bilder nebeneinander
% \usepackage{wrapfig}                                     % Fließtext um Figure-Umgebung
% \usepackage{cite}                                        % Zusatzfunktionen zum zitieren
% \usepackage{scrpage2}                                    % Wird für Kopf- und Fußzeile benötigt
% \usepackage{array,dcolumn}                               % Beide Pakete werden für die Ausrichtung der Tabellenspalten benötigt


% ############################
% # Eigene Befehle einbinden #
% ############################

% Eigene Befehle eignen sich gut um Abkürzungen für lange Befehle zu erstellen. Die Syntax ist folgende:
% \newcommand{neuer Befahl}{ein langer Befehl}
% Das folgende Beispiel fügt ein Bild mit bestimmten vorgegebenen Optionen ein:
\newcommand{\cImage}[1]{
    \begin{figure}[h!]
        \centering
        \includegraphics[width=0.50\textwidth]{#1}
    \end{figure}
}
% #1 ist dabei ein Parameter, den man \cImage übergeben muss. In 10_Titelseite.tex wird dieser Befehl verwendet. Der Parameter ist dort Bilder/titelseite.jpg.
% Benötigt man keine Parameter, dann lässt man [1] weg. Werden zusätzliche Parameter benötigt, dann kann man die Zahl auf maximal 9 erhöhen.


% #########################
% # Variablen importieren #
% #########################

% Der Befehl \newcommand kann auch benutzt werden um Variablen zu definieren:

% Nummer laut Praktikumsheft:
    \newcommand{\varNum}{XY}
% Name laut Praktikumsheft:
    \newcommand{\varName}{Vorlage}
% Datum der Durchführung:
    \newcommand{\varDate}{\today}
% Autoren des Protokolls:
    \newcommand{\varAutor}{Michael Epping}
% Nummer der eigenen Gruppe (z.B. "1mo"):
    \newcommand{\varGruppe}{Gruppennummer}
% E-Mail-Adressen der Autoren:
    \newcommand{\varEmail}{michael.epping@uni-muenster.de}
% E-Mail-Adresse anzeigen (true/false):
    \newcommand{\varZeigeEmail}{true}
% Literaturverzeichnis anzeigen (true/false):
    \newcommand{\varZeigeLiteraturverzeichnis}{true}
% Stil der Einträge im Literaturverzeichnis
    \newcommand{\varLiteraturLayout}{unsrtdin}

\newboolean{show}

% #########################
% # Beginn des Dokumentes #
% #########################

\begin{document}
\selectlanguage{ngerman}                                   % Schreibsprache Deutsch
\onehalfspacing                                            % 1 1/2 facher Zeilenabstand
\addtokomafont{sectioning}{\rmfamily}                      % Schriftsatz
\numberwithin{equation}{section}                           % Nummerierung der Formeln entsprechend der Section (z.B. 1.1)
\addtokomafont{caption}{\small\linespread{1}\selectfont}   % Ändert Schriftgröße und Zeilenabstand bei captions

% Römische Ziffern als Seitenzahlen für Titelseite bis einschließlich dem Inhaltsverzeichnis
\setcounter{page}{1}
\pagenumbering{roman}

% #######################################
% # Kopf- und Fußzeile konfigurieren    #
% #######################################

\ihead{\textit{\varNum\ - \varName }}                      % Innenseite der Kopfzeile
\chead{}                                                   % Mitte der Kopfzeile
\ohead{\textit{\varAutor}}                                 % Außenseite der Kopfzeile
\ifoot{}                                                   % Innnenseite der Fußzeile
\cfoot{- \textit{\pagemark} -}                             % Mitte der Fußzeile
\ofoot{}                                                   % Aussenseite der Fußzeile

% ###################################
% # Ausrichtung der Tabellenspalten #
% ###################################

\newcolumntype{,}[1]{D{,}{,}{#1}}                          % , in Tabellen untereinander stellen
\newcolumntype{p}{D{p}{\pm}{-1}}                           % +- in Tabellen untereinander stellen

% ########################
% # Titelseite einbinden #
% ########################

\begin{titlepage}
    \vspace*{4cm}
    \begin{center}
        \Huge
        \textbf{\varName}\\
        \vspace{1cm}
        \large
        Protokoll zum Versuch Nummer {\varNum} vom \varDate \\
        \vspace{2,5cm}
        \IfFileExists{Bilder/titelseite.png}{
            \cImage{Bilder/titelseite.png}
        } % Nach \IfFileExists muss eine Leerzeile eingefügt werden

        \vspace{1,5cm}
        \varAutor \\  
        \vspace{1cm}
        \normalsize
        \textit{\varGruppe} \\
        \newboolean{showEmail}
        \setboolean{showEmail}{\varZeigeEmail}
        \ifthenelse{\boolean{showEmail}}{\textit{\varEmail}\\}{}  
    \end{center}
\end{titlepage}


% ################################
% # Inhaltsverzeichnis einbinden #
% ################################

\tableofcontents
\newpage

% Zurücksetzen der Seitenzahlen auf arabische Ziffern
\setcounter{page}{1}
\pagenumbering{arabic}

\pagestyle{scrheadings}                                    % Ab hier mit Kopf- und Fußzeile

% ###################################
% # Den Inhalt der Arbeit einbinden #
% ###################################

\section{Einleitung}
In diesem Versuch geht es um die spektrale Zerlegung verschiedener Lichtquellen. Anhand des Spektrums können Aussagen über die Lichtquellen gemacht werden.
Bei einem Spektrometer wird zunächst mit Hilfe einer Linse ein paralleles Lichtbündel hergestellt. Dieses wird durch ein Prisma aufgrund der Wellenlängenabhängigkeit des Brechungsindexes in die verschiedenen Wellenlängen aufgeteilt. Diese Lichtstrahlen gehen durch eine zweite Linse und werden nach Wellenlängen auf einen Schirm fokussiert.
Anstelle eines Prismas ist es auch möglich ein Spektrometer mit einem Beugungsgitter zu realisieren. Dieses hat einen Vorteil bei der Bestimmung der Wellenlänge. Diese kann durch das Abzählen der Beugungsordnung $ \textit{m} $ und anhand der Winkels der zugehörigen Beugungswinkel $ \vartheta_{m} $ zusammen mit der Gitterkonstanten $ \textit{g} $ folgt für die Wellenlänge,
\begin{equation}
\lambda=\frac{g\cdot\sin \vartheta_{m}}{m} \label{Wellenlänge}
\end{equation}
daraus lassen sich für unterschiedliche Lichtquellen charakteristische Spektren bestimmen ( zum Beispiel das Linienspektrum einer Natrium-Dampflampe). 
Eine Eigenschaft die zu berücksichtigen ist, ist das Auflösungsvermögen des Spektrometers.
Also wie groß der Wellenlängenunterschied $ \triangle\lambda $ sein muss, damit zwei Linien klar unterscheidbar sind.
Das Auflösungsvermögen $ \textit{A} $ ist definiert als,
\begin{equation}
A=\frac{\lambda}{\triangle\lambda_{min}}
\end{equation}
hierbei ist $ \triangle\lambda_{min} $ die noch kleinste trennbare Wellenlängendifferenz zweier Linien ist und $ \lambda $ die mittlere Wellenlänge der Linien.
Das Rayleigh-Kriterium für das Auflösungsvermögen lautet,
\begin{equation}
\frac{\lambda}{\triangle\lambda_{min}}=mN
\end{equation}
hierbei ist $ \textit{m} $ die Beugungsordnung und  $ \textit{N} $ die Anzahl der beleuchteten Spalte.
In diesem Versuch werden unter anderem die Linienspektren von Gasen untersucht.
Gase emittieren nur diskrete Spektren, da die Elektronen nur auf bestimmten Energieniveaus sein können. Werden die Elektronen des Gases angeregt (durch Hitze oder elektrischer Spannung) können diese nur auf diskrete Niveaus höherer Energie angehoben werden. Dieser Zustand ist allerdings nicht stabil, also "fällt" das Elektron wieder auf sein ursprüngliches Energieniveau zurück. Dabei kann es die überschüssige Energie in Form von Licht abgeben, dabei folgt es der Gleichung
\begin{equation}
\lambda=\frac{hc}{\triangle E}
\end{equation}
$ \triangle E $ ist die Energiedifferenz zwischen Anfangs- und Endzustand, $ \textit{h} $ ist das Planksche Wirkungsquantum, $ \textit{c} $ ist die Lichtgeschwindigkeit und $ \lambda $ die Wellenlänge des emittierten Lichts. 
Dadurch lassen sich beispielsweise Rückschlüsse auf die Zusammensetzung des Füllgases einer Energiesparlampe schließen.
Bei Leuchtdioden sind die Spektren nicht diskret, sonder sie überdecken kontinuierlich ganze Energiebereiche. Diese werden Energiebänder genannt. 
Eine Leuchtdiode emittiert Licht, wenn die Elektronen des p-dotierten Substrates sich durch Anregung (elektrische Spannung) mit den "Löchern" des n-dotierten Substrates rekombinieren. Die emittierten Wellenlängen verteilen sich kontinuierlich über das Gesamte Energieband. Es gilt,
\begin{equation}
\lambda=\frac{hc}{E_{G}}
\end{equation}
$ E_{G} $ liegt bei Halbleitern bei etwa 1-3 eV.

\newpage
\section{Auswertung}
Wird das Prisma als optische Komponente im Spektrometer verwendet und so ausgerichtet, dass der Strahlengang symmetrisch ist. Der Spalt wird mit einer  Natriumdampflampe ausgeleuchtet. Es sind wie erwartet zwei Linien sichtbar. Diese Linien waren aber nicht gerade sondern leicht gekrümmt, was auf eine nicht ganz richtige Ausrichtung des Spektrometers zurückzuführen ist.
Wird das Prisma durch das Transmissionsgitter mit  g=\SI{1/300}{\milli\meter}  ersetzt, sind Linien bis zur 4 Beugungsordnung auffindbar. Dabei ist auffällig, dass je höher die Beugungsordnung wird, desto mehr war erkennbar, dass es sich um zwei sich überschneidende Linien handelt. Diese sind nicht scharf unterscheidbar.
Wird das Transmissuinsgitter mit einem ausgetauscht, dessen Gitterkonstante  g=\SI{1/600}{\milli\meter}  so ist zu beobachten, dass die Hauptmaxima weiter auseinander gehen ( Abstände verdoppeln sich) und das die Intensität stärker abnimmt.

% ####################
% # Anhang einbinden #
% ####################

% Löscht man die Datei "`20_Anhang.tex"', dann wird kein Anhang erzeugt.
\IfFileExists{20_Anhang}{
    \newpage
    \appendix
    \section{Anhang}
    \label{anhang}

\subsection{Fehlerrechnung}

%Links zu den Dokumentationen der verwendeten Pakete.
%
%\begin{itemize}
%    \item inputenc: \url{http://ctan.org/pkg/inputenc}
%    \item fontenc: \url{http://ctan.org/pkg/fontenc}
%    \item babel: \url{http://ctan.org/pkg/babel}
%    \item amsmath: \url{http://ctan.org/pkg/amsmath}
%    \item ifthenx: \url{http://ctan.org/pkg/ifthenx}
%    \item graphicx: \url{http://ctan.org/pkg/graphicx}
%    \item units: \url{http://ctan.org/pkg/units}
%    \item setspace: \url{http://ctan.org/pkg/setspace}
%    \item hyperref: \url{http://ctan.org/pkg/hyperref}
%    \item caption: \url{http://ctan.org/pkg/caption}
%    \item subfig: \url{http://ctan.org/pkg/subfig}
%    \item wrapfig: \url{http://ctan.org/pkg/wrapfig}
%    \item cite: \url{http://ctan.org/pkg/cite}
%    \item scrpage2: \url{http://www.komascript.de/komascriptbestandteile}
%    \item array: \url{http://ctan.org/pkg/array}
%    \item dcolumn: \url{http://ctan.org/pkg/dcolumn}
%\end{itemize}
%
%Leider wird nicht für jedes Pakete eine Dokumentation angeboten.

} % Nach \IfFileExists muss eine Leerzeile eingefügt werden

% ###################################
% # Literaturverzeichnis mit BibTeX #
% ###################################

\setboolean{show}{\varZeigeLiteraturverzeichnis}
\ifthenelse{\boolean{show}}{
    \newpage
    \bibliography{literatur}
    \bibliographystyle{\varLiteraturLayout}
}{}

% #######################
% # Ende des Dokumentes #
% #######################

\end{document}
