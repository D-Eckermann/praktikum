\section{Einleitung}
In diesem Versuch sollen optische Abbildungen mit Hilfe einer digitalen Spiegelreflexkamera untersucht werden.
\subsection{Optische Abbildungen}
Für dünne Linsen und achsennahe Strahlen gilt die Linsengleichung,
\begin{equation}
\frac{1}{f}=\frac{1}{b}+\frac{1}{g}
\end{equation}
\textit{g} ist der Abstand des Gegenstands zu Linse, \textit{b} ist der Abstand der Bildebene zur Linse und \textit{f} bezeichnet die Brennweite.
Aus dem Quotienten von \textit{b} und \textit{g} lässt sich der Abbildungsmaßstab \textit{M} bestimmen.
\begin{equation}
M=\frac{b}{g}
\end{equation}
Dieser ist ein Maß für die Vergrößerung bzw. die Verkleinerung des Objektes.
\subsection{Optische Auflösung und Blende}
Wegen der Wellennatur des Lichts, gibt es eine Aulösungsgrenze für optische Abbildungen. Da die Abbildungsleistung von Linsen in der Mitte am besten ist wird um die Auflösung zu verbessern eine Blende ( auch Apertur genannt) mit einem Durchmesser \textit{D} vor oder hinter die Linse gebracht. Dadurch gelangen nur noch Strahlen innerhalb des Durchmessers zur Linse. Das Licht wird durch die Blende gebeugt und es entstehen typischerweise Beugungsscheibchen. Damit zwei Bildpunkte von einander unterscheidbar sind, muss das Maximum des ersten mit dem Minimum des zweiten Bildpunktes zusammenfallen (Rayleigh-Kriterium).
Für eine kreisförmige Apertur lässt sich die Auflösung der Linse durch,
\begin{equation}
d=2,4391...\cdot\lambda\cdot\frac{f}{D}
\end{equation}
beschreiben. \textit{d} entspricht der Größe des Beugungsscheibchens für die Wellenlänge $ \lambda $.
Der Quotient von \textit{f} und \textit{D} wird auch Blendenzahl gennant und mit \textit{k} bezeichnet,
\begin{equation}
k=\frac{f}{D}
\end{equation}
neben der Rolle im Auflösungsvermögen, reguliert diese Größe ebenfalls die Lichtmenge die durch die Linse auf den Sensor fällt.
Dabei gilt,$ große Blendenzahl \Rightarrow kleine Lichtmenge $ bzw.
$ kleine Blendenzahl \Rightarrow große Lichtmenge $.
\subsection{Schärfentiefe}
Normalerweise wird nur ein Punkt Scharf auf die Bildebene abgebildet. Alle Punkte die weiter oder näher entfernt liegen, erscheinen als Zerstreuungskreise. Damit ein Punkt als scharf empfunden wird, darf der Durchmesser der Zerstreuungskreise einen Wert \textit{Z} nicht überschreiten, dieser entspricht üblicherweise 1/1500 der Bilddiagonalen.

