\section{Einleitung}
In diesem Versuch geht es darum Wellenphänomene anhand von Mikrowellen zu untersuchen.
Mikrowellen haben eine Wellenlänge von 30 \si{cm} bis 1\si{mm}, somit können Wellenphänomene die sonst mit sichtbaren Licht schwer zu beobachten sind besser beobachtbar.
\subsection{Braggsche Refexion}
Bei der Braggschen Reflexion trifft eine ebene Welle unter dem sogenannten Glanzwinkel \alpha auf ein Gitter aus (in diesem Fall) Metallkugeln. Diese sind auf mehreren Ebenen verteilt, der Abstand wird Netzebenenabstand genannt und mit \textit{d} bezeichnet.
Wenn eine Welle in ein solches Gitter einfällt so wird ein Teil von ihr an der ersten Ebene der Metallkugeln reflektiert, die nicht-reflektierte Strahlung wird an einer der folgenden Ebenen reflektiert. Also reflektiert jede Netzebene eine ebene Welle. Diese Wellen sind phasenverschoben und interferieren miteinander. Konstruktive Interferenz tritt auf wenn gilt,
\begin{equation}
2\textit{d}\sin \alpha=n\lambda
\end{equation}
also wenn der Gangunterschied zweier benachbarter Netzebenen reflektierten Wellen ein ganzzahliges Vielfaches der Wellenlänge beträgt.
\subsection{Totalreflexion und Evaneszente Welle}
Von Totalreflexion wird gesprochen, wenn ein Lichtstrahl aus einem optisch dichterem Medium in ein optisch dünneres Medium (n_{1}>n_{2}) eindringt und der transmittierte Strahl mit einem Winkel von 90° = \vartheta_{2} gebrochen wird, somit entsteht kein transmittierender Strahl.
Aus dem Brechungsgesetz,
\begin{equation}
n_{1}\sin \vartheta_{1}=n_{2}\sin\vartheta_{2}
\end{equation}
folgt mit \vartheta_{2}=90°
\begin{equation}
\sin\vartheta_{T}=\frac{n_{2}}{n_{1}}
\end{equation}
Falls \vartheta_{1}>\vartheta_{T} ist, dann wird 100% der einfallenden Welle Reflektiert.
Allerdings existiert auch im Fall der Totalreflexion eine Welle im Medium 2, diese propagiert allerdings parallel zu Grenzfläche. Diese Welle wird allerdings im Abstand von wenigen Wellenlängen vernachlässigbar klein. Diese Welle wird \textit{evaneszente} Welle genannt.
Wird nun ein drittes Medium mit n_{3}>n_{2} so nah an Medium 1 heran, dass die evaneszente Welle eine im Verhältnis große Amplitude hat, kann diese durch Medium 2 in das Medium 3 propagieren und sich dort fortsetzten. Dies wird \textit{frustrierte Totalreflexion} genannt.



\newpage
\section{Auswertung}
\subsection{Totalreflexion}
Um mit Hilfe des PVC-Halbzylinders eine Totalreflexion zu erzeugen wird der Sender so ausgerichtet, dass dieser  auf die runde Oberfläche strahlt. Diese Strahlen werden gebrochen und propagieren durch den Halbzylinder.
Wenn die Mikrowellen auf die Grenzfläche zwischen PCV-Halbzylinder und Luft treffen (n_{PVC}>n_{Luft}), wird der transmittierende Strahl total reflektiert, wenn der Sender im richtigen Winkel auf den Halbzylinder strahlt.
Wird nun der zweite PVC-Halbzylinder in die nah genug an den Ersten gebracht, so entsteht eine frustrierte Totalreflexion.
Die Stärke dieser frustrierte Totalreflexion nimmt mit dem Abstand der Halbzylinder voneinander exponentiell ab.
(hier Grafik einfügen)



\newpage
\section{Diskussion} 